\documentclass[letterpaper]{article} 
\usepackage{fancyhdr, cite, hyperref, lastpage}
\usepackage{siunitx}
\pagestyle{fancy}
\setlength{\oddsidemargin}{12pt}
\setlength{\evensidemargin}{10pt}
\setlength{\marginparwidth}{0in}
\setlength{\textwidth}{438pt}
\fancyhead{}
\fancyfoot{}
\fancyheadoffset[R]{1pt}
\fancyfootoffset[R]{1pt}
\fancyhead[R]{\small Research Statement. Aurojit Panda}
\fancyfoot[R]{\thepage/\pageref{LastPage}}
\renewcommand{\headrulewidth}{0pt}
\fancypagestyle{plain}{%
    \fancyhf{}
    \fancyfoot[R]{\thepage/\pageref{LastPage}}
    \renewcommand{\headrulewidth}{0pt}
    \renewcommand{\footrulewidth}{0pt}
    \fancyfootoffset[R]{1pt}
}

\newcommand\eat[1]{}
\begin{document}
\title{\Large Research Statement}
\author{Aurojit Panda}
\date{}
\thispagestyle{empty}
\maketitle
Originally networks were designed to be extremely resilient (i.e., they were designed to recover from arbitrary sets of
link and switch failures) but high availability was not a requirement. This meant that in practice
networks could take tens or hundreds of milliseconds (if not longer) to recover from failures, while the control plane computed  a new
set of forwarding rules. This is no longer acceptable for modern networks, which support many critical time-sensitive
applications. Therefore, high availability is a key concern for modern network design.

In addition, networks were originally seen as only providing best-effort packet delivery. This assumptions is also no
longer true, as most networks support a broader set of functionality. Much of this additional functionality is provided by the deployment of middleboxes, which are
used by the network operator to improve network performance and reliability. For instance, ISPs use a combination of web
proxies (for caching static content), traffic shapers, and WAN optimizers to increase perceived end user performance for
networks. They also deploy a variety of stateful and application-specific firewalls to provide isolation between
customers, restrict traffic according to legal requirements, and to protect the network from denial of service and other
attacks. The number and variety of deployed middleboxes~\cite{sherry2012making} has increased greatly over the past
years as operators have discovered both new opportunities for functionality and new threats against which networks must be
protected. Furthermore, there has been a recent trend in industry for moving middlebox functionality into software
network functions (NFs). This allows network operators to consolidate several middleboxes into a single physical server
running several NFs (with each NF isolated in a virtual machine), improves the rate at which new functionality can be deployed and simplifies management.

A third recent development is that carriers, which once merely provided network connectivity as their product, are now
interested in providing a ``service platform'' on which new services can be built. That is, carriers are are looking for
ways to support the deployment of content delivery, IPTV services, VoIP services, to be offered either by themselves, or by third parties.

\eat{Secondly, customers (both end user and content providers) expect networks to be highly available: in particular
requiring little or no disruption during failures. Providing high availability means that networks can no longer wait
for control plane convergence (at least not for tens or hundreds of milliseconds) to restore connectivity (i.e. ensure
that there are paths from a source to a destination). Furthermore, any misconfiguration of middleboxes or switches can
be catastrophic, requiring that customers wait not just for control plane convergence but for human intervention. Modern
networks therefore need to protect against a wide variety of failure scenarios, including not just equipment failure but
also misconfiguration.}

These three developments have inspired my recent research agenda. Specifically:
\begin{itemize}
    \item I have worked on routing techniques that move the responsibility for basic connectivity to the data plane,
        allowing the network to recover on packet time scales and providing high availability. While the data plane
        provides connectivity (which we argue requires only local information), route optimization (i.e., finding shortest paths) continues to be the
        control plane's responsibility.
    \item High availability applies not just to packet delivery but also to the whole range of functions provided by
        networks (for instance isolation). I am currently developing tools to verify the correctness of networks with
        dynamic path elements such as middleboxes. Previous verification techniques focus only on static data path
        elements such as routers and switches.
    \item I have worked on primitives for service-level virtualization that can help carriers provide platforms
        for deploying new network services. 
    \item Finally, I am now working on a framework, ZCSI, that leverages static and dynamic checks from programming languages
    to provide software isolation for network functions. This would allow multiple NFs to run on a system, without
    requiring each NF to run on a separate process, or VM. This both allows more fine-grained scheduling (since the OS
    scheduler is no longer responsible for decisions on what packets are processed next), better performance and enable
    denser NF deployments (since we avoid the overhead imposed by VMs). Furthermore, we are also exploring ways to allow
    NFs to be written in high-level languages without negatively impacting packet processing performance. The use of
    high-level languages can both improve developer productivity, and enable a wider set of developers to build NFs.
    This might even allow the NF landscape to evolve so it is similar to today's cloud infrastructure, with third party
    application developers deploying new, novel network functions.
\end{itemize}

\section*{Network Resilience}
Traditionally in networking we make a distinction between the data plane and the control plane: the data plane forwards
packets based on local state (e.g., the contents of a router's FIB), the control plane establishes this forwarding state
through distributed algorithms or manual configuration. In a naive implementation of this two-plane approach, the
network can recover from failure only after the control plan has computed a new set of paths and updated the state on
all routers. However the disparity in timescales between data plane operations (where a packet can be forwarded in less
than a microsecond) and control plane convergence (which can require several hundred milliseconds) mean that failures
often lead to unacceptably long outages.

To reduce the length of these outages, the control plane is often tasked with precomputing and installing an additional
set of failover paths in the data plane. When a failure is detected the data plane utilizes these precomputed paths to
forward packets. However, these failover techniques require careful configuration (for instance MPLS FRR requires that
operators set up backup tunnels after accounting for the probability of correlated failures, etc.) and provide no
guarantees on failover. Several recent failures~\cite{sprint2006, emea2008, wiki2012} have been attributed to link
failures despite the use of failover paths. 

This begs the question: can we automatically precompute a set of static paths that can provide connectivity in the
presence of any set of failures (that do not physically partition the network). In~\cite{feigenbaum12resilience} we
showed that finding a static configuration that could respond to any set of failures was impossible in general. Given
this impossibility result it is clear that just precomputing failovers is not sufficient.

Our algorithm, Data Driven Connectivity (DDC)~\cite{liu13ddc} uses simple changes in forwarding state predicated
only on the destination address and the input port of a packet to ensure that forwarding choices always direct packets
to their destination in the absence of physical partitions (packets might still be dropped due to congestion or
corruption). The changes in forwarding state can be implement in ASICs, allowing networks using DDC to achieve perfect
connectivity entirely in the data plane.

DDC however does not provide any bounds on path length, nor does it account for congestion or guarantee in-order
delivery. These issues are instead addressed by the control plane which can rely on global information and work at much
slower timescales. The DDC paradigm therefore separates connectivity (i.e., correctness for the forwarding state) from
routing (i.e. efficient forwarding) and shows that one can achieve the former with purely local information. 

Separating connectivity from efficient routing also allows for more complex control plane algorithms: the data
plane can now establish basic connectivity and the control plane can utilize this connectivity to exchange messages and
run distributed algorithms that restore efficiency.

We have analyzed the DDC algorithm extensively through simulation and software implementation, and are interested in incorporating it into switches (either at the ASIC level, or in the management software) so that we can test it on operational networks.

\section*{Network Verification}
DDC provides maximal resilience against switch and link failures in a network. Misconfiguration might however still result in
the network behaving incorrectly: for instance a misconfigured firewall might not provide required isolation properties.
Previous work like Header Space Analysis (HSA)~\cite{kazemian2012header} and Veriflow~\cite{khurshid13veriflow} have
focused on verifying the correctness of forwarding configuration: i.e., they can be used to analyze forwarding behavior
given control plane routing configuration.

These works take as input the set of routing tables (the RIB) in switches and infer the network transfer function, i.e.,
the set of paths traversed by different classes of packets through the network. These techniques rely on a few
assumptions to make this inference, in particular they assume: (1) forwarding behavior is only affected by the control
plane's configuration of the RIB, (2) forwarding behavior depends only on the packet header (as a result they do not model or track the packet
payload) and (3) forwarding is performed by ASICs which support a limited set of operations. The last assumption in
particular is important in making their approach tractable.

Unfortunately middleboxes and other network elements violate all three of these assumptions: (1) the forwarding behavior
of learning switches, stateful firewalls, etc. depends not only on their configuration but also on past traffic, (2)
application specific firewalls and other middleboxes often operate at layer 4 or layer 7 and their behavior depends on
the packet payload and (3) middleboxes are commonly built using general purpose CPUs and do not have the semantic
limitations of ASICs. Thus, these newly developed verification techniques (e.g., Veriflow and HSA) cannot be
used~\cite{panda2015new}. But such global analysis techniques are necessary because the overall behavior of the
network is a function not just of the configuration of an individual middlebox but depends on the configuration of all
middleboxes traversed. For instance the presence of a WAN optimizer might affect the behavior of an application
specific firewall, especially if the WAN optimizer appears before the firewall.

My current research focuses on building tools and techniques for proving the correctness of networks containing
middleboxes and other dynamic network element. Our approach builds on model checking which has been recently
used~\cite{dobrescu2014software} to verify correctness of individual middleboxes. Model
checking~\cite{jhala2009software, han2007providing} is a commonly used formal verification technique where correctness
is proved using a mathematical model for the program.  Naive model checking scales poorly with increase in a program's
state space and much of the work on model checking focuses on reducing the state space that needs to be explored.

At a high level our approach is to treat each middlebox as a subroutine and the network as a large program composed of
many of these subroutines. Network policy (specified using mechanisms like SIMPLE~\cite{qazi2013simple}) specifies
the order in which each subroutine is called for a specific type of packet. Network invariants are then specified as
requirements that either some packets always trigger certain subroutines, some packets never reach certain end hosts
(i.e., a subroutine is never called with inputs of a certain kind), etc. Decomposing middleboxes and network policy in
this manner allows middlebox models to be written (or generated) independently and reused for different networks.

My research has been devoted to making this approach scalable. I am focusing on two main advances: (i) minimizing the level of detail in the model, so that is only captures those aspects relevant to network behavior, and (ii) finding correctness conditions that allow invariant checking to no require a full-network analysis.

\begin{itemize}
\item Simplifying models:  A naive implementation of our high level approach might depend on fully specified models for middleboxes, for instance
models generated by a tool like Klee~\cite{cadar2008klee}. Middlebox code can however be complex both for packet
processing (for instance WAN optimizers include code for the compression logic) and for providing administrative
interfaces, etc. Further as models are combined, model complexity grows exponentially, and the use of fully specified
models renders model checking intractable even for reasonably small networks.

However, global network properties do not require fully specified models. For instance, while other
network elements might be affected by the observation that a WAN optimizer changes the packet payload they are
ambivalent to the precise transformation applied (except perhaps to track things like a pair of functions
where one compresses and the other expands are inverses of each other). This  means that models do not need to precisely
define the mechanism by which packets are transformed or classified and can instead just abstractly specify what parts of a
packet are transformed (a WAN optimizer changes the payload) or the possible results of a classification operation (an
application firewall can mark a packet as either benign or harmful). Furthermore parts of the code not run in
the data path can be entirely elided from the models. This abstraction significantly reduces model complexity making
verification tractable for many networks.

\item Eliminating global analysis:
Unfortunately the previous optimization is not sufficient for all networks. Large networks can have
several thousands or tens of thousands of middleboxes and even with simplified models verification is intractable. However, even
in these large networks an individual packet (or a class of packets) only traverses a few middleboxes.  

Limiting the model to only the few middleboxes in a path means that the complexity of the network model is a function of
the longest path between a source destination pair, rather than the size of the network and makes verification tractable
for almost all real world network. Unfortunately limiting the model to only the middleboxes in a path requires that
their behavior be independent of packets sent along any other (possibly intersecting) path. We call this property ``path
independence'' and find that it is not true in general. For instance the contents cached in a web proxy are affected by
all end hosts who can reach the proxy. However, even in these cases, it is often possible to find a smaller subset of
the network to analyze: this is based on the observation that network topologies are largely symmetric and that 
network policy is often expressed in terms of policy groups, each of which contains several end hosts. Based on these
observations, one can often analyze a smaller network containing just one host from each policy group when verifying
network reachability invariants. My current work looks at how to automatically find such a subnetwork for a given
network and invariant.

\end{itemize}

\section*{Service Level Virtualization}
I am also interested in primitives that can be provided by network operators to support the developers of
networked applications. Many existing applications, for instance file hosting services like Dropbox, P2P
applications and others build and use ad hoc notions of locality (i.e., some metric on network distance) to optimize
file transfer times. Each of these applications invents new error-prone mechanisms to discover locality. The network
operator is in a better position to provide this information (since they have access to the network topology) and can
allow application developers to leverage this information to build richer and better applications.

The idea that the network can provide additional information to improve application performance or decrease complexity
has been proposed before (for instance by the ALTO project~\cite{seedorf2009traffic}). Previous proposals however
suffered from two major drawbacks: (i) they were only applicable to certain applications and certain protocols, (ii) they
required that network administrators place new hardware within the network. We argue that the second limitation is now
gone: as a result of NFV, large ISP networks have commodity server hardware deployed within
their networks. Furthermore ISPs and network providers want to offer additional services as a means to gain new revenue
streams and differentiate themselves from the competition.

Our research asks what is a minimal set of reusable primitives that can be implemented (by network operators) and used
by a wide range of applications or third parties (referred to as tenants henceforth). By focusing on reusable primitives
we make sure that we are not tied to particular services or protocols and in fact are usable in a wide range of
settings. We call this service level virtualization since we envision that the network operator can instantiate virtual
copies of these services on behalf of tenants.

The current set of primitives that we support include: (i) local discovery: a DNS like service that provides a
set of local (i.e. nearby) peers, (ii) global discovery: a DNS like per application service that provides
tenants with the ability to rapidly change the information served by the service, and (iii) name service
demultiplexing: allows for the coexistence of different naming schemes (for instance ICN like content naming
schemes and DNS like names) and sends requests to different naming services based on the type of query, etc.
and others.

We have shown that by using these simple primitives, tenants can provide a wide variety of network applications --- such as
VPNs, caching, file sharing services, etc. These primitives are accessible through a simple interface, and their use
greatly reduces code complexity and the barrier to deploying new applications for services, and thus speeds up the speed
at which they can be developed.

Our current prototype builds five primitives (including the three listed above) on top of a software-defined network
with software switches at the end. The network administrator is responsible for creating new instances of the service
(which is similar to deploying new software on a server). Tenants then have access to both service itself (which is
accessible by sending packets or flows to a particular address) and a coordinator through which these services can be
configured. Tenant applications can then send requests to these services instead of relying on backend servers. Tenants
can configure how packets are forwarded among these various services (for instance if a query cannot be resolved by the
local discovery service forward it to global discovery, etc.). 

We have used the flexibility offered by these primitives and the configuration mechanism to build a number of prototype
applications, including a file hosting application in a few 100 lines of C++ code. By contrast writing a similar
applications in the current internet would require building mechanisms for discovering locality, building cache and
storage servers, etc. and would therefore require orders of magnitude more code.  

\section*{Framework for Building Network Function}
The current NFV deployment model envisions that network operators deploy network functions (NFs) in virtual
machines or containers, which are connected by virtual switches within the same machine. This deployment model ensures isolation
between network functions running on the same physics server, guaranteeing correctness when mutually-untrusted NFs are place on
the same physical machine. The use of VMs and containers however comes at a cost: transferring data and control between these involves
going through the OS scheduler which can add $0.5 -- 1$\si{\micro\second} of latency per context switch and the use of virtual
machines adds significant resource overheads. Furthermore, while the overhead is lower for containers, it is not 
negligible.

While these costs have not been an issue in other contexts, NFV workloads must forward packets at very high-rates (14
million packets per second or higher) to be able to saturate modern network links. In this environment, cross-VM
communication is common, and meeting latency requirements (which require that packets be processed within several
\si{\micro\second}) requires frequent context switches. My current research proposes an alternate framework, ZCSI, which
provides the same level of isolation and safety as the existing VM based solutions while eliminating many of the
overheads associated with this isolation.

Previous work on the Singularity~\cite{hunt2007singularity} operating system has shown that static and runtime checks 
provided by managed languages like C\# and Java can be leveraged to provide the same level of isolation as provided by
processes. ZCSI leverages this isolation mechanism and provides a framework where all NFs in a physical server run in
a single address space, while retaining the isolation provided by container based techniques. This eliminates the
context switch and resource overheads indicated above while still providing isolation. Furthermore, since cross-VM
packet transfers are common in NFV platforms, we have developed novel static and runtime checks that allow us to provide
safe zero-copy packet transfers within this framework.

Moving to a framework such as ZCSI can provide benefits beyond just performance. ZCSI provides a convenient framework
for exploring scheduling policies for NFs, and can be used to provide performance and resource isolation in addition to
memory isolation. Additionally, NFs are currently written in C, and are heavily optimized to achieve high
packets-per-second performance. This reliance on low-level optimization greatly increases the time for developing an NF,
and also results in such code being fragile. Network functions in ZCIS are written using high-level languages, and make use of
framework provided data structures and operations that are specifically tuned to meet the requirements of NFs, this
both reduces the time taken to develop NFs, provides improved performance and eliminates many common classes of bugs.
Finally, many existing NFs are supplied as compiled binaries, which are optimized (by developers and compilers) to
specific hardware microarchitectures. These NFs cannot easily take advantage of new CPU features, and do not perform as
well on other microarchitectures (e.g., across different generations of Intel processors). ZCSI's use of managed
languages and framework supplied data structures allow us to optimize NF performance for specific processors: we rely on
JITting to generate low level code that is optimized to the specific micro-architecture being used, thus greatly
improving performance. Furthermore, JITing also allows NFs written using ZCSI to be portable, i.e., they can run on
entirely different CPU architectures. 

Currently we have an early prototype of ZCSI, my current work focuses on building a more full-featured framework, which
can support a wide range of NFs. Once we have a basic framework, we will begin investigating abstractions that allow a
wide range of NFs to be implemented efficiently on top of ZCSI, and then investigate questions around performance
isolation, scheduling, etc.

\section*{Summary}
Overall my research focuses on building tools and techniques to respond to evolving network demands. It is essential
that networks provide a wider set of more easily usable services (and enable new applications) but this cannot be at the
cost of availability and correctness. At the same time a highly available but mostly stagnant network is not acceptable
either. Therefore it is important to balance both these concerns and build tools that help with both.
\bibliographystyle{abbrv}
\bibliography{bibs}
\end{document}
