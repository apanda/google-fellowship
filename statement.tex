\documentclass[letterpaper]{article} 
\usepackage{fancyhdr, cite}
\pagestyle{fancy}
\rhead{\small Research Statement. Aurojit Panda}
\lhead{}
\renewcommand{\headrulewidth}{0pt}
\newcommand\eat[1]{}
\begin{document}
\title{\Large Research Statement}
\author{Aurojit Panda}
\date{}
\thispagestyle{empty}
\maketitle
Networks provide higher level services and functionality beyond packet forwarding. Such services are used for
a variety of reasons including as a means to support protocols (for instance session border controllers for
VoIP sessions), to improve network performance (for instance WAN optimizers), shape traffic based on
application, provide isolation among tenants, provide caches (both in the guise of CDNs and as web caches for
wireless service). These services have traditionally been implemented in middleboxes, which in turn are built
using commodity server hardware running specialized software. This use of a common hardware platform allows
network providers to rapidly deploy newer services. Network Function Virtualization (NFV) has recently emerged
as an industry standard and allows operators to deploy new service instances easily (often using the same
tools web services use to elastically deploy new services).  This shift means network operators can allow
customers (other companies) to deploy services on their network.  Recent trends like ISP provided CDNs, the
large computation power available in IXP datacenters, etc. seem to point to a trend where carrier networks
become available as a platform.

At the same time networks continue to be critical infrastructure and network operators must ensure the
correctness of their networks. Traditionally the network data plane and control plane were tightly coupled and
network infrastructure evolved relatively slowly, allowing network operators to exhaustively test software and
hardware before deploying it on their network. More recently Software Defined Networking (SDN) has
allowed the network control plane to evolve independently of the data plane. This in conjunction with the
trend towards NFV has increased the need for quick, automatic verification of network behavior. Recent
work~\cite{khurshid13veriflow, kazemian2012header} has looked at verifying the correctness of forwarding
tables (e.g., ensuring that the network is loop free, etc.). This work however does not account for dynamic
data path behavior exhibited by many middleboxes and hence provides an incomplete analysis for many networks.

My recent research has focused on the intersection of both these trends. I have been looking at applying
techniques from the Programming Languages literature to verify administrator provided invariants (for instance
isolation, ensuring traversal, etc.) for networks with dynamic elements like middleboxes. Some of my other
research has separately focused on the architecture and design of primitives that application developers can
use to take advantage of network operator provided services. These primitives allow application developers to
query the network for additional information and to effectively use services instantiated by the network
operator.

\section*{Network Verification}
SDN has enabled the modularization of network control planes providing software with a well defined API for
affecting the data plane. Modularization has resulted in centralization which allows software to observe
forwarding state across the network. Simultaneously the use of a common API allows software to reason about
the semantics of the network data plane based upon this forwarding state. Recent work like Header Space
Analysis (HSA)~\cite{kazemian2012header} and Veriflow~\cite{khurshid13veriflow} have used these properties to
verify forwarding state and provide tools to reason about forwarding paths. 

Both HSA and Veriflow can be used to verify the network transfer function, i.e., the paths traversed by
packets given the set of forwarding tables installed in the data path. These techniques however have a few
important limitations: 1. They assume that forwarding state is only modified by the control plane, i.e.,
data plane behavior is only affected by control plane actions. This assumption is violated by several network
elements; for instance, learning switches and stateful firewalls depend on previous data path traffic. 2.
HSA can handle changes to the layer 3 (and layer 2) headers of a packet but does not account for modifications to the
packet payload. Middleboxes are often application specific and can be affected by changes to the packet
payload. 3. The specific techniques used are tractable in part because of the limited set of operations provided by
router ASICs. In particular it is not tractable to extend a similar form of reasoning to general purpose
packet processors of the kind used in middleboxes.

Recent studies~\cite{sherry2012making} show that enterprise networks (including extremely large networks with
over $100,000$ hosts) on average contain as many middleboxes as layer 3 routers. Similarly non-forwarding
boxes are widely used in carrier networks, for tasks such as caching, deep-packet inspection, providing
customer isolation, etc. Packets in these networks commonly pass through several dynamic elements and overall
network behavior depends on the result of processing on all of these elements. Furthermore, these elements do
not act in isolation: a WAN optimizer can adversely affect the behavior of an application level firewall, even
in the absence of configuration errors. Given this observation networks with middleboxes have at least three
additional sources of potential errors: middleboxes could be misconfigured, the network policy which decides
the order in which middleboxes are traversed might violate network invariants or a particular collection of
middleboxes might be unable to enforce a given invariant.

Concurrently other work~\cite{dobrescu2014software} has proposed using formal verification techniques to
verify correctness of a single software dataplane element, for instance a middlebox. This work relies on model
checking, an approach to formal verification where one generates a mathematical model for the program, i.e.,
the set of states the program can be in and transitions between those states. Given this model one can prove
or disprove correctness by systematically exploring all possible states and checking if a disallowed state is
ever reached. Model checking has been well studied in the programming language
literature~\cite{jhala2009software, han2007providing} and has been used to show correctness for a variety of
programs~\cite{cadar2008klee, klein2009sel4}.  Model checking however scales poorly with increase in state
space and the verification problem is intractable for general programs. As a result much of the work in this
area focuses on applying model checking to specific domains where the state space can be more efficiently
explored. 

My research focuses on extending model checking so that network wide properties can be verified even in the
presence of complex data path elements like middleboxes. Model checking is appealing in this case since it
places no constraints on middlebox semantics. 

At a high level our approach is to treat each middlebox as a subroutine and the network as one large program
composed of many of these subroutines. End hosts in this model are treated as subroutines which can send or
receive packets.  Network policy (for specified using mechanisms like SIMPLE~\cite{qazi2013simple}) specifies
the order in which each subroutine is called for a specific type of packet. Network invariants are then
specified as requirements that either some packets always trigger certain subroutines, some packets never
reach certain end hosts (i.e., a subroutine is never called with inputs of a certain kind), etc. Amongst other
things this approach allows a wide variety of invariants to be checked, allows for the decomposition of
network policy and middlebox configurations and allows each middlebox model to be written independently.
However it also presents a scalability challenge.

A naive implementation of our high level approach might depend on fully specified models for middleboxes, for
instance models generated by a tool like Klee~\cite{cadar2008klee}. Some middleboxes, for instance WAN
optimizers or application firewalls, can have complicated logic and result in models with a very large number
of variables and a large state space. Furthermore as several of these models are combined the state space
allowed by the model grows exponentially making verification intractable. Using fully specified models would
therefore render model checking intractable even for reasonably small networks.

We observe however that global properties do not require fully specified models. For instance while other
network elements might be affected by the observation that a WAN optimizer changes the packet payload they are
ambivalent to the precise transformation applied (except perhaps to track things like a pair of functions
where one compresses and the other expands are inverses of each other). This observation implies that one need
not model the precise mechanism by which complex transformations (for instance compression, encryption,
computing a hash, etc.) are carried out, or even the mechanisms by which complex decisions (behavioral
analysis marking a flow as potentially harmful) are made and can instead abstractly model what gets
transformed and possible decisions. Abstracting behavior away in this manner significantly reduces the state
space that needs to be explored (for instance a DPI box can now mark a packet as either benign or harmful and
this is the only state we see for this process rather than observing the precise transitions that led to this
decision). Such reduction also makes analysis tractable in the presence of several complex network elements.

Unfortunately the previous optimization is not sufficient for all networks. In particular large networks can
have several thousands or tens of thousands of middleboxes and even with abstract models the state space for the
entire network grows exponentially making the problem intractable. While a network might have thousands of
middleboxes, any class of packets only traverses a few of them. More importantly all packets between a
particular source destination pair only go through a few of the middleboxes in a network. 

If we could focus analysis on only the few middleboxes in a path and verify properties based on those the
worse case analysis would only require going through the states for all the middleboxes in the longest path
between a source destination pair, dramatically reducing complexity. Unfortunately such separation requires
that the properties for a path be independent of all other (possibly intersecting) paths through the network.
This is often not true: for instance the contents cached in a web proxy are affected by all end hosts whose
requests are served by the proxy. Therefore to verify an invariant that say host A can never get content from
server S we might need to verify not only the path between A and S (which goes through a web proxy W) but also
the path from all other hosts which can reach W to server S. Path independence can be restored in a number of
different ways in this case: a) placing a firewall between A and W configured to disallow requests from A to S
(and thus prevent these requests from being served from the proxy), b) disallowing the proxy from caching any
state from S or c) partitioning the cache so that A's cache is separate from other hosts (or more specifically
hosts with different policies). 

Path independence obviously makes the problem of model checking networks more tractable but is also appealing
from an administrative standpoint: since intersecting paths do not affect each other, network administrators
must only consider the paths that actually pass through a middlebox when changing middlebox configuration.
Similar when adding or removing middleboxes the only thing affected are paths on which the middleboxes appear.
Determining the exact criterion for path independence, cases in which it makes sense and generalizing these
results is the focus of my current research. Furthermore even if we cannot achieve complete path independence,
even splitting the network into independent areas also reduces the state space and might be sufficient in many
cases. 

\section*{Extended Virtualization}
Along with verifying networks which contain middleboxes and other complex network elements, other research I
am involved in investigates how application writers and the users of networks can take advantage of these
functions. 

Existing application for instance file hosting services (such as Dropbox), P2P application (such as
BitTorrent) and others rely on a notion of network distance, trying to minimize latency for network
transfers. Similarly CDNs and other content delivery services often rely on DNS or other mechanisms to direct
traffic to the nearest source for content, and VoIP providers might use servers to proxy traffic for clients
behind NATs who are unable to directly accept connections. All of these applications can be better served by
providing network primitives to accomplish these tasks: network providers know the topology of the network
connecting their customers and are in a better position provide a distance metric for applications requiring
low latency data transfer, they are often in a better position to direct traffic to caches and finally they
can instantiate proxies which are closer to the source or destination for VoIP traffic thus decreasing
latency.

The idea that the network can provide additional information to improve application performance has been
previous proposed (for instance by the ALTO project~\cite{seedorf2009traffic}). Previous proposals however
suffered from two major drawbacks: 1) they were only applicable to certain applications and certain
protocols, 2) they required that network administrators place new hardware within the network. We argue that
the second limitation is now gone: as a result of NFV and processing requirement, large ISP networks have
commodity server hardware deployed within the network. Furthermore ISPs and network providers want to offer
additional services as a means to get new revenue and differentiate themselves from the competition.

Our research in this area looks into a common set of reusable primitives that can be implemented (by network
operators) and used by a wide range of application writers (referred to as tenants henceforth). By focusing
on reusable primitives we make sure that we are not tied to particular services or protocols and in fact are
usable in a wide range of settings. We call this extended virtualization since we envision that the network
operator instantiates a virtual copy of these primitives for each tenant who makes use of these services.

The current set of primitives that we support include: i) local discovery: a DNS like service that provides a
set of local (i.e. nearby) peers, ii) global discovery: a DNS like per application service that provides
tenants with the ability to rapidly change the information served by the service, iii) name service
demultiplexing: allows for the coexistence of different naming schemes (for instance ICN like content naming
schemes and DNS like names) and sends requests to different naming services based on the type of query, etc.
and others.

Application writers can logically compose these services to provide end user services. As an example consider
an application like Dropbox. At present the Dropbox client is responsible for using DNS to discover the
origin server on which data is stored, and uses flooding or other mechanisms to discover other clients connected to
the same local area network. The client then uses this information to decide on the most efficient way to
synchronize the file system (by sending data only to the origin server, by sending data to both the origin
server and any local machines or using some other strategy). In a network supporting extended virtualization,
Dropbox could instead ask the network provider to instantiate a local and global discovery service. In this
new model Dropbox can rely on the global and local discovery services for the location of the origin server and
all nearby (as defined by the local discovery service) peers belonging to the same user, and use this
information to optimize file synchronization.

Discovering the minimal set of primitives is still an open question for this research, our current efforts
have focused on building primitives to support a small set of representative applications. As the number of
general purpose cores deployed in carrier networks increases a move towards using the network as a platform
seems inevitable.

The two research project described are synergistic since moving to a model where the network is a platform
creates a scenario where new network functionality is rapidly configured and deployed in the network. As the
pace at which new functionality is deployed increases the chance of bugs or configuration errors increases as
does the complexity of manually verifying new networking functionality. Verification therefore is an important
requirement for the network to evolve into a platform: network operators need tools to guarantee correctness
even as they allow a wider set of tenants to affect the internals of a network.

At the same time verification is interesting largely because of the pace at which networks are evolving and
because of the complexity of some of the tools deployed. As network complexity increases it is inevitable that
we will need formal verification techniques to prove correctness for these large distributed systems. 


\bibliographystyle{abbrv}
\bibliography{bibs}
\end{document}
